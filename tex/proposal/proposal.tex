\documentclass[a4paper,9pt]{article}

\usepackage[margin=0.95in]{geometry}
\usepackage{inconsolata}
\usepackage[normalem]{ulem}
% \usepackage{fontspec}
\usepackage{charter}
% \setromanfont{Times New Roman}
\usepackage[hidelinks,backref]{hyperref} % clickable links and citations with no green borders
\usepackage{amsmath}
\usepackage{listings} % add source code snippets
\usepackage{csquotes} % block quotes
\usepackage{color} % it's a must these days / for the colors are fading
\usepackage[dvips]{graphicx}
\DeclareGraphicsExtensions{.png,.jpg}
\setlength{\parindent}{0pt}
\definecolor{mygreen}{rgb}{0,0.6,0}
\definecolor{mygray}{rgb}{0.5,0.5,0.5}
\definecolor{mydarkgray}{rgb}{0.4,0.4,0.4}
\definecolor{mymauve}{rgb}{0.58,0,0.82}
\definecolor{myrust}{rgb}{0.77,0,0}
\pagestyle{empty}

\lstset{ %
  backgroundcolor=\color{white},   % choose the background color; you must add \usepackage{color} or \usepackage{xcolor}
  basicstyle=\footnotesize\ttfamily,        % the size of the fonts that are used for the code
  breakatwhitespace=false,         % sets if automatic breaks should only happen at whiterspace
  breaklines=true,                 % sets automatic line breaking
  captionpos=b,                    % sets the caption-position to bottom
  commentstyle=\color{mygreen},    % comment style
  deletekeywords={...},            % if you want to delete keywords from the given language
  escapeinside={\%*}{*)},          % if you want to add LaTeX within your code
  extendedchars=true,              % lets you use non-ASCII characters; for 8-bits encodings only, does not work with UTF-8
  frame=single,                    % adds a frame around the code
  keepspaces=true,                 % keeps spaces in text, useful for keeping indentation of code (possibly needs columns=flexible)
  keywordstyle=\color{blue},       % keyword style
  language=C++,                    % the language of the code
  morekeywords={*,...},            % if you want to add more keywords to the set
  numbers=none,                    % where to put the line-numbers; possible values are (none, left, right)
  numbersep=5pt,                   % how far the line-numbers are from the code
  numberstyle=\tiny\color{mygray}, % the style that is used for the line-numbers
  rulecolor=\color{black},         % if not set, the frame-color may be changed on line-breaks within not-black text (e.g. comments (green here))
  showspaces=false,                % show spaces everywhere adding particular underscores; it overrides 'showstringspaces'
  showstringspaces=false,          % underline spaces within strings only
  showtabs=false,                  % show tabs within strings adding particular underscores
  stepnumber=2,                    % the step between two line-numbers. If it's 1, each line will be numbered
  stringstyle=\color{mymauve},     % string literal style
  tabsize=2,                       % sets default tabsize to 2 spaces
  % title=\lstname                   % show the filename of files included with \lstinputlisting; also try caption instead of title
}

\hypersetup{
  colorlinks=true,
  linkcolor=red,
  urlcolor=blue,
  citecolor=green,
  linktoc=page
}

% \author{
%   Kandarp Khandwala\\
%   110050005
%   \and
%   Rohan Prinja\\
%   110050011
% }
% \title{Project Proposal}

\begin{document}

% \vspace*{4.6cm}

% \maketitle

\textcolor{myrust}{\Huge{\centerline{Project Proposal}}}
\textcolor{myrust}{\Large{\centerline{CS663}}}
% \vspace{16pt}

\textcolor{myrust}{\section{About}}

In this document we propose a project for the Image Processing course, CS663, and outline some of its details. \textbf{Team members}: Kandarp Khandwala (110050005) and Rohan Prinja (110050011)

\textcolor{myrust}{\section{Papers we will implement}}

We plan to implement the paper \href{http://research.microsoft.com/en-us/um/people/kopf/pixelart/}{Depixelizing Pixel Art}. The paper outlines a method for converting pixel art (very-low-resolution raster images) to vector images, and obtains results superior to simple upsampling and even high-quality existing pixel art scaling algorithms like hqx.

\textcolor{myrust}{\section{Datasets}}

The authors of the paper have demonstrated their algorithm on around 54 inputs on \href{http://research.microsoft.com/en-us/um/people/kopf/pixelart/supplementary/comparison\_bicubic.html}{this page}. We shall use these images as our dataset.

\textcolor{myrust}{\section{Validation strategy}}

To evaluate any pixel art scaling algorithm we must use visual inspection, since the aim of the paper is to present an algorithm that creates aesthetic and good-looking vector art from pixel art. Some factors to consider while visually inspecting two vector art outputs are as follows:\\

\begin{enumerate}
  \itemsep-0.25em
  \item An image with less jagged edges and less blockiness is better. For example, the image on the left is better:\\
  \includegraphics[height=5cm]{jagged}
  \item An image which is not overly smoothed is better. For example, the image on the right is better:\\
  \includegraphics[height=3cm]{corner}
  \item An image which does not have ``incorrect islands" is better. An ``incorrect island" is when an output vector art image contains a small section cut-off from a larger body even though the intention of the pixel artist was to have a single continguous body. For example, the image on the left is better:\\
  \includegraphics[height=3cm]{islands}
\end{enumerate}

We can sum up the above heuristics by saying that in general, our overall aim is to see how close the vector output is to the intention of the pixel artist. The closer it is, the better the vector output.

\textcolor{myrust}{\section{Deliverables}}

After we implement the algorithm as described in the paper, we will tune the parameters of the algorithm so as to obtain outputs better than hq4x on most or all of the input images. We do not expect to surpass the output of the paper itself, however, we will try to decrease the running time.

\end{document}